\documentclass[dvipdfmx]{jsarticle}
\begin{document}

\title{TESTTITLE}
\author{AUTHOR}
\maketitle

\hypertarget{ux6982ux8981}{%
\section{概要}\label{ux6982ux8981}}

次の公式を示す。

\begin{verbatim}
e^A B e^{-A} = B + [A,B] + \frac{1}{2}[A, [A,B]] + \cdots
\end{verbatim}

\hypertarget{ux5c0eux51fa}{%
\section{導出}\label{ux5c0eux51fa}}

次の関数をTaylor展開することで示せる。

\begin{verbatim}
f(t) = e^{tA} B e^{-tA}
\end{verbatim}

まず、一回微分 \(f'(t)\)、二階微分 \(f''(t)\) は

\begin{verbatim}
\begin{align}
f'(t) &= A e^{tA} B e^{-tA} - e^{tA} B A e^{-tA} \cr
  &= e^{tA} A B e^{-tA} - e^{tA} B A e^{-tA} \quad\because [e^{tA}, A] =0 \cr
  &= e^{tA} (AB-BA) e^{-tA} \cr
  &= e^{tA} [A,B] e^{-tA}
\end{align}
\end{verbatim}

\begin{verbatim}
\begin{align}
f''(t) &= e^{tA} A [A,B] e^{-tA} - e^{tA} [A,B] A e^{-tA} \cr
  &= e^{tA} (A [A,B] - [A,B] A) e^{-tA} \cr
  &= e^{tA} [A, [A,B]] e^{-tA}
\end{align}
\end{verbatim}

である。

ここで、交換子を \([A,B] = \mathrm{ad}_A [B]\)
と書くことにし、さらに次のように約束する。

\begin{verbatim}
\begin{align}
\mathrm{ad}_A^0 [B] &= B \cr
\mathrm{ad}_A^1 [B] &= \mathrm{ad}_A [B] = [A,B] \cr
\mathrm{ad}_A^2 [B] &= \mathrm{ad}_A [\mathrm{ad}_A [B]] = [A, [A, B]] \cr
&\vdots \cr
\mathrm{ad}_A^n [B] &= \underbrace{[A,[A,\ldots,[A}_n,B]]\cdots]
\end{align}
\end{verbatim}

この表記法を用いると、\(f'(t) = e^{tA}\\,\mathrm{ad}_A^1 [B] \\,e^{-tA}\),
\(f''(t) = e^{tA} \\,\mathrm{ad}_A^2 [B] \\,e^{-tA}\)
となる。これを一般化する。

\begin{verbatim}
\begin{align}
\frac{d}{dt} \left( e^{tA} \,\mathrm{ad}_A^n[B]\,e^{-tA} \right) &= e^{tA} A\,\mathrm{ad}_A^n[B]\, e^{-tA} - e^{tA} \,\mathrm{ad}_A^n[B]\, A e^{-tA} \cr
  &= e^{tA} \left( A\,\mathrm{ad}_A^n[B]- \mathrm{ad}_A^n[B]\,A \right) e^{-tA} \cr
  &= e^{tA} \left[ A, \mathrm{ad}_A^n[B] \right] e^{-tA} \cr
  &= e^{tA}\,\mathrm{ad}_A^{n+1}[B]\,e^{-tA}
\end{align}
\end{verbatim}

以上より、

\begin{verbatim}
f^{(n)}(t) = e^{tA} \,\mathrm{ad}_A^n[B]\,e^{-tA}
\end{verbatim}

が得られた。\(t=0\) では \(f^{(n)}(0) = \mathrm{ad}_A^n[B]\)
である。\(f(t)\) のTaylor展開により、

\begin{verbatim}
\begin{align}
f(t) &= \sum_{n=0}^\infty \frac{1}{n!} f^{(n)}(0) \, t^n \cr
  &= \sum_{n=0}^\infty \frac{1}{n!} \mathrm{ad}_A^n[B]\,t^n
\end{align}
\end{verbatim}

以上より、

\begin{verbatim}
f(0) = e^{A} B e^{-A} = \sum_{n=0}^\infty \frac{1}{n!} \mathrm{ad}_A^n[B] = B + [A,B] + \frac{1}{2}[A, [A,B]] + \cdots
\end{verbatim}

を示すことができた。

次のように書くこともできる。

\begin{verbatim}
e^{A} B e^{-A} = e^{\mathrm{ad}_A [\cdot]} B
\end{verbatim}

\hypertarget{ux4ea4ux63dbux5b50ux306eux6027ux8cea}{%
\section{交換子の性質}\label{ux4ea4ux63dbux5b50ux306eux6027ux8cea}}

定義から明らかに、超演算子 \(\mathrm{ad}_A[\cdot]\) は線形演算子である。

\begin{verbatim}
\begin{align}
\mathrm{ad}_A^n[B+C] &= [A,[A,\cdots,[A,B+C]\cdots] \cr
  &= [A,[A,\cdots,[A,B]\cdots] + [A,[A,\cdots,[A,C]\cdots] \cr
  &= \mathrm{ad}_A^n[B] + \mathrm{ad}_A^n[C]
\end{align}
\end{verbatim}

演算子 \(A\) が Normal operator
だと仮定すると、次のように固有値分解できる。

\begin{verbatim}
A = \sum_k \lambda_k \ket{k}\!\!\bra{k}
\end{verbatim}

この時、演算子 \(\ket{i}\\!\\!\bra{j}\) は超演算子
\(\mathrm{ad}_A[\cdot]\) の固有値である。

\begin{verbatim}
\begin{align}
\mathrm{ad}_A\left[\ket{i}\!\!\bra{j}\right] &= [A, \ket{i}\!\!\bra{j}] \cr
  &= \left(\sum_k \lambda_k \ket{k}\!\!\bra{k}\right) \ket{i}\!\!\bra{j} - \ket{i}\!\!\bra{j} \left( \sum_k \lambda_k \ket{k}\!\!\bra{k} \right) \cr
  &= (\lambda_i - \lambda_j) \ket{i}\!\!\bra{j}
\end{align}
\end{verbatim}

よって、\(A\) の同じ固有値に属する固有ベクトル \(\ket{i}\), \(\ket{j}\)
について、\(\ket{i}\\!\\!\bra{j}\) は
\(\mathrm{ad}_A[\ket{i}\\!\\!\bra{j}]\)
の固有演算子であり、固有値はゼロ。 \(\ket{i}\), \(\ket{j}\) が \(A\)
の異なる固有空間に属する場合、\(\ket{i}\\!\\!\bra{j}\) の固有値は
\(\lambda_i - \lambda_j\) である。

また次も成り立つ。

\begin{verbatim}
\mathrm{ad}_A^n\left[\ket{i}\!\!\bra{j}\right] = (\lambda_i - \lambda_j)^n \ket{i}\!\!\bra{j}
\end{verbatim}

同様の議論から、超演算子 \(\mathcal{I}_A[B] = e^A B e^{-A}\)
を定義するとこれも線形演算子であり、\(\ket{i}\\!\\!\bra{j}\) は固有値
\(e^{\lambda_i - \lambda_j}\) に属する固有演算子である。

\hypertarget{ux5fdcux7528}{%
\section{応用}\label{ux5fdcux7528}}

シュレディンガー表示からハイゼンベルク表示(もしくは相互作用表示)への移行を考えると、
シュレディンガー表示で演算子 \(A\) はハイゼンベルク表示で
\(A(t) = e^{iH_0t} A e^{-iH_0t}\) であるから、

\begin{verbatim}
A(t) = A + (it) [A,H_0] + \frac{(it)^2}{2!} [A, [A, H_0]] + \frac{(it)^3}{3!} [A, [A, [A, H_0]]] + \cdots
\end{verbatim}

ここで、演算子 \(\ket{i}\\!\\!\bra{j}\) が固有演算子であることを用いると

\begin{verbatim}
\begin{align}
e^A \ket{i}\!\!\bra{j} e^{-A} &= e^{\mathrm{ad}_A[\cdot]} \ket{i}\!\!\bra{j} \cr
  &= \sum_{n=0}^\infty \frac{1}{n!}\mathrm{ad}_A^n[\ket{i}\!\!\bra{j}] \cr
  &= \sum_{n=0}^\infty \frac{1}{n!} (\lambda_i - \lambda_j)^n \ket{i}\!\!\bra{j}\cr
  &= e^{\lambda_i - \lambda_j} \ket{i}\!\!\bra{j}
\end{align}
\end{verbatim}

である。

\(A \to i H_0 t\) と置き換えると、摂動ハミルトニアン
\(H_1 = [h_{ij}]_{i,j}\) の相互作用表示 \(H_1^I(t)\) は

\begin{verbatim}
H_1^I(t) = e^{iH_0t} H_1 e^{-iH_0t} = \sum_{i,j} e^{i(\epsilon_i - \epsilon_j)t} h_{ij} \ket{i}\!\!\bra{j}
\end{verbatim}

となり、\href{https://dia-pe-titech.esa.io/posts/406}{実験マニュアル・基礎理論/基礎理論/回転座標変換と相互作用表示
\#量子力学 - dia-pe-titech.esa.io} と同様の表式が得られた。
注)今の所リンク先と符号が違うけど、向こうの記事が間違ってる\ldots{}

また、エルミート演算子 \(K\) としてユニタリ演算子 \(U=\exp(iK)\)
によるユニタリ変換\(A' = UAU^\dagger\) を考えてみる。\(K\)
の固有値は全て実数であるから、\(iK\) の固有値は全て純虚数。全ての \(i\),
\(j\) について \(|e^{\lambda_i - \lambda_j}|=1\)
より、ユニタリ変換は行列要素の大きさを変えず、位相だけを変化させることがわかる。

\hypertarget{ux6240ux611f}{%
\section{所感}\label{ux6240ux611f}}

\begin{itemize}
\tightlist
\item
  おもったより面白くてキレイな関係が得られた。
\item
  指数関数で挟む=交換子を指数的に適用する という対応ができる。
\item
  交換子を \(\mathrm{ad}\)
  で表記するのはどこかで見たやり方だけど、一般的に通じるものではないと思う
\end{itemize}

\end{document}
